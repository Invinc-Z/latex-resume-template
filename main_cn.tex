% !TEX TS-program = xelatex
% !TeX encoding = UTF-8

\documentclass{resume}
\usepackage{zh_CN-Adobefonts_external} % Simplified Chinese Support using external fonts (./fonts/zh_CN-Adobe/)
% \usepackage{NotoSansSC_external}
% \usepackage{NotoSerifCJKsc_external}
%\usepackage{zh_CN-Adobefonts_internal} % Simplified Chinese Support using system fonts
\usepackage{linespacing_fix} % disable extra space before next section
\usepackage{cite}
\usepackage{color}  % 改变背景颜色
\definecolor{shadecolor}{rgb}{0.92,0.92,0.92} % 灰色
\usepackage{hyperref} % 网址超链接
\usepackage{multirow}
\usepackage{setspace}
%\setstretch{1.2} % 设置全文行距

\begin{document}
\pagenumbering{gobble} % suppress displaying page number

% ============================= 联系方式左右结构 ============================
%\begin{tabular*}{\textwidth}{@{\extracolsep{\fill}} l l}
%\multirow[c]{3}{*}[-0.05in]{\Huge\kaishu 姓名}
%    & \email{zhuangzhuangzhang97@gmail.com} \\
%    & \phone{188-8888-8888} \\
%    & \github{https://github.com/Invinc-Z} \\
%\end{tabular*}

% ============================= 联系方式左右结构带照片 ============================
%\begin{tabular*}{\textwidth}{l @{\hspace{6.8cm}} r|l}
%\multirow[c]{3}{*}[-0.05in]{\Huge\kaishu 姓名}
%&  \multirow{3}{*}{
%    \includegraphics[width=1.5cm]{avatar}
%}
%& \email{zhuangzhuangzhang97@gmail.com} \\
%& & \phone{188-8888-8888} \\
%& & \github{https://github.com/Invinc-Z} \\
%\end{tabular*}

% ============================= 联系方式上下结构 ============================
\name{\Huge  \kaishu 姓名}
\basicInfo{
  \email{zhuangzhuangzhang97@gmail.com} \textperiodcentered\ 
  \phone{188-8888-8888} \textperiodcentered\ 
  \github{https://github.com/Invinc-Z}
}
  
% ============================= 个人基本信息 ================================
\section{\faAddressCard  \fangzheng \ 基本信息}
\begin{tabular*}{\textwidth}{@{\extracolsep{\fill}} c c c c c}
    性别:男 & 年龄:22岁 & 现居地:苏州 & 工作年限:两年半 & 求职意向:C++ 软件开发工程师 \\
\end{tabular*}


% ============================= 教育背景 ====================================
\section{\faGraduationCap \fangzheng \ 教育背景}
\datedsubsection{\textbf{XXXXXX大学}~~ \ 硕士, 信息与通信工程, 排名: \nth{1}/50}{2019.09 -- 2022.06}
\datedsubsection{\textbf{XXXXXX大学}~~ \ 学士, 通信工程, 排名: 10\%}{2015.09 -- 2019.06}

% ============================= 专业技能 ====================================
\section{\faHammer \fangzheng \ 专业技能}
% increase linespacing [parsep=0.5ex]
\textbf{编程语言}:
\begin{itemize}[parsep=0.5ex]
  \item C++(熟练):掌握面向对象编程(封装/继承/多态)、熟悉STL容器使用、熟悉几种常见的设计模式、智能指针、多线程、模板编程
  \item C(熟练):熟悉内存管理、指针操作、文件I/O、数据结构及算法实现、多线程/进程、网络编程
  \item Python(熟练): ...
\end{itemize}

\textbf{系统与工具}: 
\begin{itemize}[parsep=0.5ex]
  \item 熟练掌握Linux基本命令、熟悉Vim、Makefile及GCC/GDB的使用
  \item 熟悉Linux进程管理、多进程/线程、进程池和线程池、IO多路复用模型
  \item 掌握socket网络编程、熟悉TCP/IP协议
  \item 了解SVN、Git版本控制工具的基本使用
\end{itemize} 

\textbf{数据库与缓存}:
\begin{itemize}[parsep=0.5ex]
  \item MySQL(熟练):SQL优化、索引设计、事务管理
  \item Redis(熟练):数据结构应用(String/Hash/List)、持久化、高并发场景
\end{itemize}

\textbf{其他}:    
\begin{itemize}[parsep=0.5ex]
  \item LaTeX(熟练):掌握论文排版、技术文档编写、能进行模板设计和修改、环境配置和编译报错解决
  \item Markdown(熟练):熟练使用Markdown编写技术文档、博客笔记,熟悉Pandoc格式转换
\end{itemize}

% ============================= 工作经历 ====================================
\section{\faUsers \fangzheng \ 工作经历}

\datedsubsection{\textbf{XX有限公司}~/~软件工程师}{2022.08--2025.12}
\datedsubsubsection{A项目}{2024.01--2025.12}
\begin{enumerate}[parsep=0.5ex]
	\item STAR 法则是一种常用来组织回答、复盘经历或写面试回答的结构化方法,由四部分组成:
	\item \textbf{S — Situation(情境)}:说明事件发生的背景是什么
	\item \textbf{T — Task(任务)}:你当时承担的任务或目标是什么
\end{enumerate}

\datedsubsubsection{B项目}{2022.08--2023.12}
\begin{enumerate}[parsep=0.5ex]
	\item \textbf{A — Action(行动)}:你具体采取了哪些行动或策略
	\item \textbf{R — Result(结果)}:最终取得了什么结果(最好量化)
	\item 用“背景—任务—行动—结果”的顺序,把经历讲得清晰、有逻辑、有亮点
\end{enumerate}

\section{\faLightbulb \fangzheng \ 项目经历}
\datedsubsection{\textbf{面向XXX方法研究}~/~算法负责人}{2020.02--2022.05}
\begin{enumerate}[parsep=0.5ex]
		\item 
		\item 
		\item 
\end{enumerate}

% ============================= 实习经历 ====================================
%\section{\faUserTie \fangzheng \ 实习经历}

% ============================= 校内经历 ====================================
%\section{\faSchool \fangzheng \ 校内经历}

% ============================= 科研成果 ====================================
\section{\faBook \fangzheng \ 科研成果}
% increase linespacing [parsep=0.5ex]
\begin{enumerate}[parsep=0.5ex]
	\renewcommand{\labelenumi}{[\theenumi]}
	\item XXXXXXX XXXXXXXXXXX XXX  XXXXX XXXXXXXXXXXXXXX[J]. XXXXXXX.\quad \textcolor[gray]{0.5}{第一作者}
	\item 一种XXXXXXXXXXXXXXXXX的XX方法[P]. 江苏省: ZL 2021 1 0XXXXXX.5
	\item METHOD FOR XXXXXX XXXX XXXX AND XXXX XXXX IN XXXXX-XXXXX XXXXXX XXXXX.\quad PCT国际专利\quad 公开号:WO/2023/XXXXXX
	\item 一种基于XXXX的XXXXXXXXX度量方法[P]. 江苏省:ZL 2021 1 0XXXXXX.7
	\item 基于XXXX和XXXXXX的XX系统[J]. 计算机XXXXX.
\end{enumerate}

% ============================= 奖项荣誉 ====================================
\section{\faTrophy \fangzheng \ 奖项荣誉}
\datedline{2021--2022年XXXXXXXX优秀毕业研究生}{2022--06}
\datedline{2021--2022年XXXXXXXX三好研究生}{2022--06}

% ============================= 自我评价 ====================================
\section{\faPerson \fangzheng \ 自我评价}
% increase linespacing [parsep=0.5ex]
\begin{enumerate}[parsep=0.5ex]
	\item 
	\item 
	\item
\end{enumerate}

% ============================= 兴趣爱好 ====================================
\section{\faIcons \fangzheng \ 兴趣爱好}
% increase linespacing [parsep=0.5ex]
\begin{itemize}[parsep=0.5ex]
	\item \textbf{唱}:大家好,我是练习时长两年半的个人练习生Invinc-Z,喜欢唱。
	
	\item \textbf{跳}:大家好,我是练习时长两年半的个人练习生Invinc-Z,喜欢跳。
	
	\item \textbf{rap}:大家好,我是练习时长两年半的个人练习生Invinc-Z,喜欢rap。
	
	\item \textbf{篮球}:大家好,我是练习时长两年半的个人练习生Invinc-Z,喜欢篮球。
\end{itemize}

% ============================= 其他 ====================================
\section{\faInfo \fangzheng \ 其他}
% increase linespacing [parsep=0.5ex]
\begin{itemize}[parsep=0.5ex]
  \item 技术博客: \url{https://invinc-z.com/} \quad \url{https://www.cnblogs.com/Invinc-Z}
  \item GitHub: \url{https://github.com/Invinc-Z}
  \item 资格证书:...
  \item ...
\end{itemize}

%% Reference
%\newpage
%\bibliographystyle{IEEETran}
%\bibliography{mycite}

\end{document}
